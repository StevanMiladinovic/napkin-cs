\chapter{Sales pitches}\label{ch:sales}
\newcommand{\pitch}[1]{\ii[\textsf{\color{blue}\ref{#1}}.] \textsf{\color{blue} \textbf{\nameref{#1}.}} \\[1ex]} % for now. . .
\newcommand{\buzzword}[1]{\textbf{\color{green!40!black} #1}}

This chapter contains a pitch for each part,
to help you decide what you want to read
and to elaborate more on how they are interconnected.

% TODO:
%For convenience, here is again the dependency plot
%that appeared in the frontmatter.
%\input{tex/frontmatter/digraph}

\section{Mathematics}
\begin{itemize}
  \pitch{part:foundations}
  This is the fundamental amount of mathematics required to understand
  anything in this book. Which is not a lot, really. It largely concerns
  basic set theory and the nature of logical statements.

  \pitch{part:maths}
  This contains all further mathematics you are likely to encounter over the
  course of a computer science degree. Some of it (such as linear algebra) may
  be extremely general and applicable to many different areas, some of it (such
  as numerical analysis) may be essential to how computers work, other
  parts (such as probability or number theory) may be specific to certain
  applications only.
\end{itemize}

\section{Theory}
\begin{itemize}
  \pitch{part:theoretical}
  This section concerns itself with what it means to
  \emph{compute} anything at all, what a computer even is, in the theoretical
  sense, what it can and --- perhaps more importantly --- \emph{cannot} compute.
  In this, we treat computation not as a list of instructions or an interaction
  of hardware components, but as a theoretical process that follows certain
  formalisms and rules.

  We begin with finite automata and their languages --- the regular expressions,
  before moving on to context-free grammars and push-down automata and
  finally into Turing machines, which model all of what a real computer
  can do.

  We will also delve into proofs
  of the correctness of such machines, systems verification, formal process
  algebra and more.

  \pitch{part:dsa}
  This is where we move away from the question of
  \emph{whether} something can be computed and instead ask ourselves
  \emph{how} and \emph{how well} it can be computed, as well as how to
  actually do so in the most efficient way possible.

  We begin with the basics of how data may be represented in memory, or in a more
  abstract context, using various data structures.

  Then we'll introduce algorithms to traverse, sort, search and manage these
  structures, all while measuring how efficient various approaches and
  algorithms are in terms of time and memory required.

  We analyze why, how and in which context certain approaches are more efficient
  than others and how you may perform such analysis on your own, for any
  algorithm, theoretical or implemented.

  \pitch{part:architecture}
  At its core, a computer is really just an electronic machine, in which
  electrons flow in a certain way to react to inputs and produce outputs
  via their currents.

  In this section, we concern ourselves with how to construct a CPU, how
  memory is structured and accessed, and how performance is impacted by
  both physical and logical design, as well as their interactions.

  We will also briefly introduce machine language and assembly code,
  and cover how digital files get loaded into the CPU and transformed into
  physical effects, as well as how modern assembly techniques like pipelining
  may improve performance.

  In essence, this section covers the intersection of, and various means of
  interfacing between hardware and software.

  \pitch{part:quantum}
  What happens when a computer operates by the laws of quantum physics?
  Here, we shall journey into the strange world of quantum computation, where
  bits may be zero or one\dots or something in between.
  Where reading data may be a write-operation and all notions of
  \emph{parallelism} fundamentally change meaning.

  We begin by introducing the required physics --- just enough to understand
  what \emph{superposition}, \emph{entanglement} and \emph{interference} are,
  before building up the mathematics required to represent quantum states.

  The core of this part of the book focuses on quantum circuits and how they
  are used to model and design algorithms for quantum computers. We will solve
  a number of conventionally \emph{hard} problems, such as prime factorization
  (using Shor's algorithm) or searching (using Grover's algorithm) in a
  radically more efficient manner than conventional computers would allow.
\end{itemize}

\section{Software Development}
\begin{itemize}
  \pitch{part:programming}
  Programming is the primary tool used to \emph{make computers do things}.
  In this chapter, we will cover the basics of what types of languages are used,
  which paradigms they are used in and what styles we choose and why. Here, we
  cover all aspects of programming from low-level to abstract.

  We begin with imperative programming: variables, control-flow, loops, state
  and mutation. Then we move on to object-oriented programming, and see how
  encapsulation, inheritance and polymorphism affect larger software systems.

  Beyond that, we will cover the functional programming paradigm, in which
  code consists of pure functions, in the mathematical sense and side-effects
  are avoided, as well as logic programming, where computation turns into a
  matter of proof and constraint satisfaction.

  We will finish this part with practical tips on how to learn the process of
  programming itself, new programming languages and how to work in unfamiliar
  paradigms, before giving a list of useful and educational practice problems
  to learn or master a variety of subjects, from a programming language itself
  to various highly theoretical or practical applications covered in this very
  book.

  \pitch{part:engineering}
  Software engineering is all the parts of the development process of software
  that \emph{aren't} programming. From designing software to meet requirements,
  to planning the development of large projects and testing, maintaining and
  managing existing software.

  We begin with core design patterns, which serve as reusable solutions to
  common architectural problems, then cover how to gather requirements,
  write specifications, perform version control and coordinate changes safely
  over time, even across large groups.

  Testing and debugging will be a major theme: how do you ensure that software
  does precisely what you want it to do, and keeps doing so as you extend it?
  We will also cover agile methods, project management best-practices and DevOps
  methodology --- covering deployment, monitoring and CI/CD.\@

  \pitch{part:hci}
  Good software should not just be functional, technically solving a problem.
  It should also be intuitively usable. This part is about how to properly
  design systems for human use, ensuring that our intended user base may
  understand, control and benefit from the clever algorithms and programs we
  come up with.

  We begin with principles of usability and accessibility: how to create
  interfaces that are clear, efficient and inclusive, before exploring visual
  design, interaction modeling and feedback loops, using desktop, mobile and
  web-interfaces as examples.

  We also study how to evaluate an interface, process user-feedback and other
  metrics, as well as how to perform proper user testing and iterative
  refinement.
\end{itemize}

\section{Systems and Infrastructure}
\begin{itemize}
  \pitch{part:compilers}
  We've heard about assembly at this point.
  But how does high-level code, utilizing complex abstractions, turn into
  byte-wise instructions for our CPU\@?

  We begin with lexical analysis (turning text into abstract tokens),
  parsing (representing the structure of a program), and intermediate
  representations, before exploring various optimizations that compilers
  employ to make code run faster without changing its meaning, and finally
  to code-generation, in which assembly instructions are emitted for a processor
  to physically execute.

  We also delve into type systems that can catch errors before the code is ever
  executed, and runtime systems that manage such things as memory, garbage
  collection and function calls.

  Building a compiler is a process that involves theory, algorithms and
  low-level systems architecture. After reading this part of the book, you
  should have understood a core part of every programming language, and perhaps
  even be able to create your own.

  \pitch{part:db}
  As we store more data than could (or should) fit into a single data structure,
  we need a database. This part of the book introduces the theory and practice
  of managing, organizing, structuring and maintaining data at scale.

  We begin with the concept of relational databases and SQL, the Standard Query
  Language used to access and modify structured data. Then we will study how to
  design schemas, normalize tables and use Entity-Relation diagrams to model
  real-world domains.

  Transactions, indexing and concurrency follow --- giving you the tools to
  understand \emph{how} databases keep data consistent across failures, respond
  quickly to any query and scale to several million users in the most extreme
  cases.

  Towards the end of this section, we then cover NoSQL models and how one may
  structure data in other manners, as well as the motivation for doing so.

  \pitch{part:os}
  How does a CPU decide what to run? How are multiple processes running at once,
  without interfering with each other? How is memory allocated without conflicts
  and why can two programs seemingly access the same addresses?

  Here, we answer those questions and more, by delving into the design of
  operating systems. We will study the nature of processes and threads, CPU
  and I/O scheduling, memory management, file systems and virtualization.

  We will explore how operating systems enforce the boundaries between
  processes while allowing for inter-process communication, and how they
  optimize performance on modern hardware. We will also cover classic
  problems like deadlocks, race-conditions and resource contention, as well
  as how they are solved in theory and practice.

  \pitch{part:distributed}
  Not all computer processes run on a single machine.
  This part shall cover those systems which run \emph{in parallel}, whether
  that is on multiple processor cores, or across multiple physically separate
  machines.

  We begin by covering concurrency and multithreading: how one designs a program
  that can do multiple things at once, safely. Then we move to distributed
  algorithms and the concepts of consensus, fault-tolerance and coordination
  across unreliable networks (which are all networks).

  We study protocols like gossip and Paxos, as well as systems like MapReduce
  and Spark, and explore modern architectures for compute clusters, cloud
  computing and data processing at scale.

  \pitch{part:networking}
  Computers are more useful when connected to other computers.
  Networking is how they are connected, and in this part of the book we will
  cover how that works, from the physical signals traveling through wires up to
  the final application.

  We begin with the OSI model and TCP/IP protocol, laying out the architecture
  of the internet. Then we look at how data is addressed, routed and
  transmitted, and how sockets allow programs to communicate with one another.

  We'll also cover common protocols (like HTTP, HTTPS, FTP, DNS), network
  performance analysis and network-related security concerns and threats
  and their mitigations.

  \pitch{part:cybersec}
  Once your computer connects to others --- or to an untrusted user --- security
  is no longer optional. Here, we cover how systems may be attacked by a
  malicious actor and how they may be defended against such attacks.

  We begin by covering cryptography: how data is encrypted and signed and
  identities are verified, before exploring authentication systems, common
  vulnerabilities (like overflow and injection attacks), thread modeling and
  security policies.

  Later chapters will cover blockchains, secure communication protocols and the
  human factors of digital security.
\end{itemize}

\section{Applications}
\begin{itemize}
  \pitch{part:ai}
  How do computers learn from data, recognize patterns or reason about uncertain
  scenarios? This part of the book shall introduce the tools and ideas behind
  machine learning and AI\@.

  We begin with statistical learning --- fitting models to given data --- and
  proceed through classification, clustering and regression algorithms using
  neural networks and random forests.

  Towards the end of this part, we explore the theory behind large language
  models and natural language processing, to see how modern generative AI
  creates texts or summarizes documents.

  \pitch{part:graphics}
  How do you turn data into images? How do you simulate the appearance of light,
  color and motion on a screen? In this section, we use computation to generate
  and manipulate visual scenes.

  We will cover the core rendering pipeline, from modeling geometry to
  projecting a 2D or 3D scene onto a screen. We will introduce shaders,
  ray-tracing and texture mapping, as well as covering human perception,
  color theory and visual design.
  In later chapters, we will touch on augmented and virtual reality.

  \pitch{part:bioinformatics}
  Life sciences --- meaning biology, chemistry and related fields --- have
  become a data science of their own. In this part, we explore how computation
  may help us model the properties of substances and the processes of their
  interactions, as well as help us understand life itself.

  We begin with sequence analysis: how to align DNA, determine similarity or
  distance between two biological sequences, find motifs and compare individuals and
  species based on their DNA\@.

  We will then explore computer-based structure analysis and chemoinformatics,
  in which computers may help design drugs, interpret experiments and uncover
  the rules that govern complex chemical interactions.

  Finally, we will cover computational systems biology, in which we use methods
  of computer science to interpret large datasets regarding biological and
  medical applications.
\end{itemize}
