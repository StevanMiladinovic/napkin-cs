\chapter{Mathematical Statements and Formal Logic}\label{ch:statements}
\section{Definition of a Statement}

\begin{definition}
  A statement (in the mathematical or logical sense) can always be sensibly
  assigned a truth value, namely ``true'' or ``false'' (also represented as 1/0).
  Statements the truth of which we have accepted by proof may be called
  \emph{theorem}, \emph{lemma}, \emph{proposition} or \emph{corollary},
  depending on their importance.
  Statements which we assume to be true by definition alone are called
  \emph{axioms}.
\end{definition}

\begin{example}
  [Examples and counterexamples of mathematical statements]
  \begin{itemize}
    \leavevmode
    \ii{} ``\(1+2=3\)'' is a true statement.
    \ii{} ``Squares of natural numbers may be negative'' is a false statement
          (but \emph{it is} a statement!).
    \ii{} ``x+1=2'' only becomes a mathematical statement when we choose a
          particular value of \(x\), or otherwise add something along the lines
          of ``there exists a natural number x, such that''. On its own, it is
          not a statement.
   \ii{} ``This statement is false'' is not a mathematical statement at all,
          since it can be neither true nor false.
  \end{itemize}
\end{example}

\section{Logical Operations}
Once we have constructed a logical statement, we may apply operations to it.
Logical statements may be combined freely with these operators, always producing
other logical statements. We define the following common operations by their
truth tables:

{ % Block for local table labeling
\captionsetup[table]{labelformat=empty, hypcap=false}

\begin{tcolorbox}[colframe=gray!80, colback=white, boxrule=0.4pt, top=4pt, bottom=4pt]
  \begin{minipage}[t]{0.32\textwidth}
    \centering
    \vspace{-10pt} % reduce space above caption
    \captionof{table}{Negation \(\neg\) \\(\emph{not A})}
    \vspace{10pt}  % add space between caption and table
    \begin{tabular}{c|c}
      \(A\) & \(\neg A\) \\
      \hline
      0 & 1 \\
      1 & 0
    \end{tabular}
  \end{minipage}
  \hfill
  \begin{minipage}[t]{0.32\textwidth}
    \centering
    \vspace{-10pt}
    \captionof{table}{Conjunction \(\land\) \\(\emph{A and B})}
    \vspace{3pt}
    \begin{tabular}{cc|c}
      \(A\) & \(B\) & \(A \land B\)\\
      \hline
      0 & 0 & 0 \\
      0 & 1 & 0 \\
      1 & 0 & 0 \\
      1 & 1 & 1 \\
    \end{tabular}
  \end{minipage}
  \hfill
  \begin{minipage}[t]{0.32\textwidth}
    \centering
    \vspace{-10pt}
    \captionof{table}{Disjunction \(\lor\) \\(\emph{A or B})}
    \vspace{3pt}
    \begin{tabular}{cc|c}
      \(A\) & \(B\) & \(A \land B\)\\
      \hline
      0 & 0 & 0 \\
      0 & 1 & 1 \\
      1 & 0 & 1 \\
      1 & 1 & 1 \\
    \end{tabular}
  \end{minipage}
\end{tcolorbox}

\begin{remark}
  These tables read (for two example rows) as follows: \\
  If logical statement \(A\) is true, then \(\neg A\) is a logical statement
  meaning \emph{not A} and it is false.\\
  If logical statement \(A\) is false and logical statement \(B\)
  is false, then \(A \land B\) is a logical statement meaning
  \emph{A and B} and it is false.
\end{remark}

Perhaps somewhat more interesting for our purposes are the following logical
operations

\begin{tcolorbox}[colframe=gray!80, colback=white, boxrule=0.4pt, top=4pt, bottom=4pt]
  \begin{minipage}[t]{0.48\textwidth}
    \centering
    \vspace{-10pt}
    \captionof{table}{Implication \(\Rightarrow\) \\(\emph{if A then B})}
    \vspace{3pt}
    \begin{tabular}{cc|c}
      \(A\) & \(B\) & \(A \Rightarrow B\)\\
      \hline
      0 & 0 & 1 \\
      0 & 1 & 1 \\
      1 & 0 & 0 \\
      1 & 1 & 1 \\
    \end{tabular}
  \end{minipage}
  \hfill
  \begin{minipage}[t]{0.48\textwidth}
    \centering
    \vspace{-10pt}
    \captionof{table}{Equivalence \(\Leftrightarrow\) \\
      (\emph{A is equivalent to B})}
    \vspace{3pt}
    \begin{tabular}{cc|c}
      \(A\) & \(B\) & \(A \Leftrightarrow B\)\\
      \hline
      0 & 0 & 1 \\
      0 & 1 & 0 \\
      1 & 0 & 0 \\
      1 & 1 & 1 \\
    \end{tabular}
  \end{minipage}
\end{tcolorbox}
}

Which are often used to express the flow of formal proofs.
\(A \Rightarrow B\) reads as \emph{if A then B}. Meaning that if statement \(A\)
is true, \(B\) must always also be true. Note that this makes no statement
about the value of \(B\) in the case where \(A\) is false.
\(A \Leftrightarrow B\) on the other hand reads as \emph{A if and only if B}.
Either both are true, or neither are. In some sense, they are fundamentally the
same statement put in different terms.

A less common form of notation is \(A \Leftarrow B\) which simply reads as
\emph{if B then A} and is equivalent to \(B \Rightarrow A\).

\section{Tautologies}

We can, using these methods, construct statements that are always true.
A simple example of this might be \(A \lor \neg A\) or \(A \Rightarrow A\),
where \(A\) is any statement. Similarly, we may also construct statements that
must always be false, such as \(A \land A\) or \(A \Leftrightarrow \neg A\).

We call statements that are always true by nature \emph{tautologies} and
those that are always false \emph{contradictions}. They are denoted by the
symbols \(\top\) and \(\bot\), respectively.

\subsection{Compound Statements}
We may freely combine logical statements using any of these operations, always
creating new logical statements at each step. For instance,
\begin{example}
  [Compound statements]
  \((A \lor B) \Rightarrow C\)
  \hfill
  \((A \land B) \Leftrightarrow C\)
  \hfill
  \(A \lor (B \land C)\)
  \hfill
  \(A \land (B \Leftarrow C)\)
\end{example}

are all logical statements that may be true or false, depending on the values of
\(A, B, C\).
I have used parentheses to clarify the intended order of operations here, which
is often done for simplicity. However, just like with arithmetic operations
(\(+, -, \times, \div\)), precedence rules still exist:
\begin{itemize}
  \ii{} Negation \(\neg\) is evaluated first, followed by
  \ii{} Conjunction \(\land\)
  \ii{} Disjunction \(\lor\)
  \ii{} Implication \(\Rightarrow\) and
  \ii{} Equivalence \(\Leftrightarrow\)
\end{itemize}

\begin{example}
  [Signifying the precedence of logical operators]
  Consider the compound logical statement:\\
  \begin{center}
    \(\neg A \lor A \land B\)
  \end{center}
  It reads as follows: Either \(A\) is false, or \(B\) and \(C\) are both true.

  This is correctly evaluated by first negating \(A\), then evaluating
  \(B \land C\), then the logical or combining both.
\end{example}

\section{\problemhead}
\begin{problem}
  [Statements]
  Which of the following are \emph{mathematical statements}?
  \begin{enumerate}
    \ii{\(3+5=8\)}
    \ii{\(12-2=15\)}
    \ii{\(x+3=10\)}
    \ii{Let \(x\) be an integer.}
    \ii{There exists a prime number greater than \(1000\).}
    \ii{This sentence is false.}
    \ii{Every even number greater than \(2\) is the sum of two primes.}
  \end{enumerate}

  \begin{hint}
    Any mathematical statement must be either definitely true or definitely
    false. But you don't necessarily have to \emph{know} whether it is true or
    false. There exist statements the truth of which is yet to be discovered.
  \end{hint}

  \begin{sol}
  \begin{enumerate}
    \ii{\(3+5=8\)} We know for a fact that this is true. So it is a statement.
    \ii{\(12-2=15\)} We know for a fact that this is false. So it is a statement.
    \ii{\(x+3=10\)} This is not a fully specified statement. It may be true or
          false, depending on the value of \(x\). On its own, it is not a statement.
    \ii{Let \(x\) be an integer.} This is a declaration or an instruction. Not a
          statement in the mathematical sense. It has no truth value associated
          with it.
    \ii{There exists a prime number greater than \(1000\).} This is true. So it
          is a statement.
    \ii{This sentence is false.} This sentence is paradoxical. It cannot be
          true, nor can it be false. Hence, it is not a valid statement.
    \ii{Every even number greater than \(2\) is the sum of two primes.}
          This is a statement. We do not know if it is true (it's a famous
          unsolved problem, in fact!), but we know for a fact that it must be
          either true or false. So it satisfies the definition of what a
          statement is.
  \end{enumerate}
  \end{sol}
\end{problem}

\begin{problem}
  [Truth tables]
  Construct the truth-table for the following statement:\\
  \begin{center}
    \(A \Rightarrow (B \lor \neg C)\)
  \end{center}

  \begin{hint}
    First compute the truth tables for the intermediate values of \(\neg C\) and
    \(B \lor \neg C\).
  \end{hint}

  \begin{sol}
    \begin{tabular}{ccc|cc|c}
      \(A\) & \(B\) & \(C\) & \(\neg C\) & \(B \lor \neg C\) &
      \(A \Rightarrow (B \lor \neg C)\)\\
      \hline
      0 & 0 & 0 & 1 & 1 & 1\\
      0 & 0 & 1 & 0 & 0 & 1\\
      0 & 1 & 0 & 1 & 1 & 1\\
      0 & 1 & 1 & 0 & 1 & 1\\
      1 & 0 & 0 & 1 & 1 & 1\\
      1 & 0 & 1 & 0 & 0 & 0\\
      1 & 1 & 0 & 1 & 1 & 1\\
      1 & 1 & 1 & 0 & 1 & 1\\
    \end{tabular}
  \end{sol}
\end{problem}

\begin{problem}
  [Logical Operator Precedence]
  Using parentheses, make the order of operations explicit, without changing the
  meaning of the statement.\\
  \(\neg A \lor B \land C \Rightarrow D\)

  \begin{hint}
    \begin{itemize}
      \ii{} Negation \(\neg\) is evaluated first, followed by
      \ii{} Conjunction \(\land\)
      \ii{} Disjunction \(\lor\)
      \ii{} Implication \(\Rightarrow\) and
      \ii{} Equivalence \(\Leftrightarrow\)
    \end{itemize}
  \end{hint}

  \begin{sol}
    \(((\neg A) \lor (B \land C)) \Rightarrow D\)
  \end{sol}
\end{problem}

\begin{problem}
  [Logically Evaluating Statements]
  Let \(A=0, B=1\). Evaluate the following statement:\\
  \begin{center}
    \((\neg A \lor B) \Leftrightarrow (B \land \top)\)
  \end{center}

  \begin{hint}
    Evaluate both sides of the equivalence first, then apply the equivalence relation.
  \end{hint}

  \begin{sol}
    \begin{align*}
      (\neg A \lor B) &\Leftrightarrow (B \land \top)\\
      (\neg 0 \lor 1) &\Leftrightarrow (1 \land 1)\\
      (1 \lor 1) &\Leftrightarrow (1)\\
      1 &\Leftrightarrow 1\\
      1
    \end{align*}
  \end{sol}
\end{problem}

\begin{problem}
  [Tautology and Contradiction]
  Is the following statement a tautology, a contradiction or neither?
  Can any statement be both?\\
  \((A \Rightarrow B) \Leftrightarrow (\neg A \lor B)\)

  \begin{sol}
    The statement is a tautology. It expresses a common logical identity, namely
    that \(A \Rightarrow B\) is the same statement as \(\neg A \lor B\).\\
    No statement can be both tautology and contradiction, since that would
    require it to be true and false simultaneously. This would contradict our
    definition of what a logical statement is.
  \end{sol}
\end{problem}
