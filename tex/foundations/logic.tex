\chapter{Mathematical Statements and Formal Logic}\label{ch:statements}
\section{Definition of a Statement}

\begin{definition}
  A statement (in the mathematical or logical sense) can always be sensibly
  assigned a truth value, namely ``true'' or ``false'' (also represented as 1/0,
  or \(\top\)/\(\bot\)).
  Statements the truth of which we have accepted by proof may be called
  \emph{theorem}, \emph{lemma}, \emph{proposition} or \emph{corollary},
  depending on their importance.
  Statements which we assume to be true by definition alone are called
  \emph{axioms}.
\end{definition}

\begin{example}
  [Examples and counterexamples of mathematical statements]
  \begin{itemize}
    \leavevmode
    \ii{} ``\(1+2=3\)'' is a true statement.
    \ii{} ``Squares of natural numbers may be negative'' is a false statement
          (but \emph{it is} a statement!).
    \ii{} ``x+1=2'' only becomes a mathematical statement when we choose a
          particular value of \(x\), or otherwise add something along the lines
          of ``there exists a natural number x, such that''. On its own, it is
          not a statement.
   \ii{} ``This statement is false'' is not a mathematical statement at all,
          since it can be neither true nor false.
  \end{itemize}
\end{example}

\section{Logical Operations}
Once we have constructed a logical statement, we may apply operations to it.
Logical statements may be combined freely with these operators, always producing
other logical statements. We define the following common operations by their
truth tables:

{ % Block for local table labeling
\captionsetup[table]{labelformat=empty, hypcap=false}

\begin{tcolorbox}[colframe=gray!80, colback=white, boxrule=0.4pt, top=4pt, bottom=4pt]
  \begin{minipage}[t]{0.32\textwidth}
    \centering
    \vspace{-10pt} % reduce space above caption
    \captionof{table}{Negation \(\neg\) \\(\emph{not})}
    \vspace{3pt}  % add space between caption and table
    \begin{tabular}{c|c}
      \(A\) & \(\neg A\) \\
      \hline
      1 & 0 \\
      0 & 1
    \end{tabular}
  \end{minipage}
  \hfill
  \begin{minipage}[t]{0.32\textwidth}
    \centering
    \vspace{-10pt}
    \captionof{table}{Conjunction \(\land\) \\(\emph{and})}
    \vspace{3pt}
    \begin{tabular}{cc|c}
      \(A\) & \(B\) & \(A \land B\)\\
      \hline
      1 & 1 & 1 \\
      1 & 0 & 0 \\
      0 & 1 & 0 \\
      0 & 0 & 0 \\
    \end{tabular}
  \end{minipage}
  \hfill
  \begin{minipage}[t]{0.32\textwidth}
    \centering
    \vspace{-10pt}
    \captionof{table}{Disjunction \(\lor\) \\(\emph{or})}
    \vspace{3pt}
    \begin{tabular}{cc|c}
      \(A\) & \(B\) & \(A \land B\)\\
      \hline
      1 & 1 & 1 \\
      1 & 0 & 1 \\
      0 & 1 & 1 \\
      0 & 0 & 0 \\
    \end{tabular}
  \end{minipage}
\end{tcolorbox}
}
