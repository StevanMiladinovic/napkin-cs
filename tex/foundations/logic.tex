\chapter{Mathematical Statements and Formal Logic}\label{ch:statements}
\section{Definition of a Statement}

\begin{definition}
  A statement (in the mathematical or logical sense) can always be sensibly
  assigned a truth value, namely ``true'' or ``false'' (also represented as 1/0,
  or \(\top\)/\(\bot\)).
  Statements the truth of which we have accepted by proof may be called
  \emph{theorem}, \emph{lemma}, \emph{proposition} or \emph{corollary},
  depending on their importance.
  Statements which we assume to be true by definition alone are called
  \emph{axioms}.
\end{definition}

\begin{example}
  [Examples and counterexamples of mathematical statements]
  \begin{itemize}
    \leavevmode
    \ii{} ``\(1+2=3\)'' is a true statement.
    \ii{} ``Squares of natural numbers may be negative'' is a false statement
          (but \emph{it is} a statement!).
    \ii{} ``x+1=2'' only becomes a mathematical statement when we choose a
          particular value of \(x\), or otherwise add something along the lines
          of ``there exists a natural number x, such that''. On its own, it is
          not a statement.
   \ii{} ``This statement is false'' is not a mathematical statement at all,
          since it can be neither true nor false.
  \end{itemize}
\end{example}

\section{Logical Operations}
Once we have constructed a logical statement, we may apply operations to it.
Logical statements may be combined freely with these operators, always producing
other logical statements. We define the following common operations by their
truth tables:

{ % Block for local table labeling
\captionsetup[table]{labelformat=empty, hypcap=false}

\begin{tcolorbox}[colframe=gray!80, colback=white, boxrule=0.4pt, top=4pt, bottom=4pt]
  \begin{minipage}[t]{0.32\textwidth}
    \centering
    \vspace{-10pt} % reduce space above caption
    \captionof{table}{Negation \(\neg\) \\(\emph{not A})}
    \vspace{10pt}  % add space between caption and table
    \begin{tabular}{c|c}
      \(A\) & \(\neg A\) \\
      \hline
      0 & 1 \\
      1 & 0
    \end{tabular}
  \end{minipage}
  \hfill
  \begin{minipage}[t]{0.32\textwidth}
    \centering
    \vspace{-10pt}
    \captionof{table}{Conjunction \(\land\) \\(\emph{A and B})}
    \vspace{3pt}
    \begin{tabular}{cc|c}
      \(A\) & \(B\) & \(A \land B\)\\
      \hline
      0 & 0 & 0 \\
      0 & 1 & 0 \\
      1 & 0 & 0 \\
      1 & 1 & 1 \\
    \end{tabular}
  \end{minipage}
  \hfill
  \begin{minipage}[t]{0.32\textwidth}
    \centering
    \vspace{-10pt}
    \captionof{table}{Disjunction \(\lor\) \\(\emph{A or B})}
    \vspace{3pt}
    \begin{tabular}{cc|c}
      \(A\) & \(B\) & \(A \land B\)\\
      \hline
      0 & 0 & 0 \\
      0 & 1 & 1 \\
      1 & 0 & 1 \\
      1 & 1 & 1 \\
    \end{tabular}
  \end{minipage}
\end{tcolorbox}

\begin{remark}
  These tables read (for two example rows) as follows: \\
  If logical statement \(A\) is true, then \(\neg A\) is a logical statement
  meaning \emph{not A} and it is false.\\
  If logical statement \(A\) is false and logical statement \(B\)
  is false, then \(A \land B\) is a logical statement meaning
  \emph{A and B} and it is false.
\end{remark}

Perhaps somewhat more interesting for our purposes are the following logical
operations

\begin{tcolorbox}[colframe=gray!80, colback=white, boxrule=0.4pt, top=4pt, bottom=4pt]
  \begin{minipage}[t]{0.48\textwidth}
    \centering
    \vspace{-10pt}
    \captionof{table}{Implication \(\Rightarrow\) \\(\emph{if A then B})}
    \vspace{3pt}
    \begin{tabular}{cc|c}
      \(A\) & \(B\) & \(A \Rightarrow B\)\\
      \hline
      0 & 0 & 1 \\
      0 & 1 & 1 \\
      1 & 0 & 0 \\
      1 & 1 & 1 \\
    \end{tabular}
  \end{minipage}
  \hfill
  \begin{minipage}[t]{0.48\textwidth}
    \centering
    \vspace{-10pt}
    \captionof{table}{Equivalence \(\Leftrightarrow\) \\
      (\emph{A is equivalent to B})}
    \vspace{3pt}
    \begin{tabular}{cc|c}
      \(A\) & \(B\) & \(A \Leftrightarrow B\)\\
      \hline
      0 & 0 & 1 \\
      0 & 1 & 0 \\
      1 & 0 & 0 \\
      1 & 1 & 1 \\
    \end{tabular}
  \end{minipage}
\end{tcolorbox}
}

Which are often used to express the flow of formal proofs.
\(A \Rightarrow B\) reads as \emph{if A then B}. Meaning that if statement \(A\)
is true, \(B\) must always also be true. Note that this makes no statement
about the value of \(B\) in the case where \(A\) is false.
\(A \Leftrightarrow B\) on the other hand reads as \emph{A if and only if B}.
Either both are true, or neither are. In some sense, they are fundamentally the
same statement put in different terms.

A less common form of notation is \(A \Leftarrow B\) which simply reads as
\emph{if B then A} and is equivalent to \(B \Rightarrow A\).

\subsection{Compound Statements}
We may freely combine logical statements using any of these operations, always
creating new logical statements at each step. For instance,
\begin{example}
  \((A \lor B) \Rightarrow C\)
  \hfill
  \((A \land B) \Leftrightarrow C\)
  \hfill
  \(A \lor (B \land C)\)
  \hfill
  \(A \land (B \Leftarrow C)\)
\end{example}

are all logical statements that may be true or false, depending on the values of
\(A, B, C\).
I have used parentheses to clarify the intended order of operations here, which
is often done for simplicity. However, just like with arithmetic operations
(\(+, -, \times, \div\)), precedence rules still exist:
\begin{itemize}
  \ii{} Negation \(\neg\) binds most tightly, followed by
  \ii{} Conjunction \(\land\)
  \ii{} Disjunction \(\lor\)
  \ii{} Implication \(\Rightarrow\) and
  \ii{} Equivalence \(\Leftrightarrow\)
\end{itemize}

\begin{example}
  Consider the compound logical statement:\\
  \begin{center}
    \(\neg A \lor A \land B\)
  \end{center}
  It reads as follows: Either \(A\) is false, or \(B\) and \(C\) are both true.

  This is correctly evaluated by first negating \(A\), then evaluating
  \(B \land C\), then the logical or combining both.
\end{example}
