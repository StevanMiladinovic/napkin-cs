\chapter{Relations and Functions}\label{ch:functions}
\section{Relations}

\begin{definition}
  A (binary) relation \(R\) between two sets \(A, B\) is the 3-tuple of those
  two sets and any subset of their Cartesian product:
  \begin{align*}
    R \defeq (A, B, R') & \text{where} & R' \subseteq A \times B
  \end{align*}
  Which means that a relation between two sets is specified by those two sets
  and a set of pairs, in which the first element of each pair is from one set
  and the second element of each pair is from the other.

  The set \(A\) is then called the \emph{domain} of \(R\), and \(B\) is called
  its \emph{codomain}.
\end{definition}

A relation \emph{relates} objects from one set to objects from another (or possibly the same) set.

To express that a given pair is in such a relation, we may use
any of the following notations:
\begin{center}
\begin{tabular}{c c}
  \((a, b) \in R\) & \text{explicit set notation}\\
  \(R(a,b)\) & \text{prefix notation}\\
  \(a R b\) & \text{infix notation}
\end{tabular}
\end{center}

\begin{example}
  [Example of a relation]
  Let \(A=\NN, B=\NN\).
  Then \(<\) defines a relation from \(A\) to \(B\).\\
  For instance, \((1, 2) \in <\), or as this is more commonly denoted:
  \(1 < 2\). \\
  However, \((1, 0) \notin <\), meaning \( 1 \not< 0\).
\end{example}

\subsection{Uniqueness Properties}

A relation is called \emph{left-unique} if every element in its domain
is related to \emph{at most one} element of its codomain.
Similarly, a relation is called \emph{right-unique} if every element in the
codomain is related to \emph{at most one} element in the domain.

\subsection{Totality Properties}

A relation is called \emph{left-total} if every element in its domain
is related to \emph{at least one} element of its codomain.
Again, the term \emph{right-unique} is used to describe the same idea
the other way around: every element of the codomain of a right-unique relation
is related to \emph{at least one} element of its domain.

\begin{tikzpicture}

% Y coordinates for 3 points each side
\def\domainY{1.5,0,-1.5}
\def\codomainY{1.5,0,-1.5}

% 1. Left-total + Right-unique (e.g., function)
\ovalrelation{0cm}{0cm}
  {1.5,0,-1.5}
  {1.5,0,-1.5}
  {1/1,2/2,3/3}

% 2. Left-unique only
\ovalrelation{7cm}{0cm}
  {1.5,0,-1.5}
  {1.5,0,-1.5}
  {1/1,2/2}

% 3. No properties
\ovalrelation{0cm}{-5cm}
  {1.5,0,-1.5}
  {1.5,0,-1.5}
  {1/1,1/2,2/1}

% 4. Function but not RT (some codomain unused)
\ovalrelation{7cm}{-5cm}
  {1.5,0,-1.5}
  {1.5,0,-1.5}
  {1/1,2/2,3/2}

% 5. RT + LU only
\ovalrelation{0cm}{-10cm}
  {1.5,0,-1.5}
  {1.5,0,-1.5}
  {1/1,2/2,2/3}

\end{tikzpicture}

\section{Functions}

\begin{definition}
  A \emph{function} is a relation that is right-unique and left-total.
  We usually denote this as
  \begin{align*}
    f \colon A \to B
  \end{align*}
  meaning \emph{``Let \(f\) be a function from \(A\) to \(B\)''}.
  This means that a function assigns each \(a \in A\) \emph{exactly one}
  \(b \in B\), though it may repeatedly assign the same elements of \(B\),
  or not assign some elements from \(B\) to any elements in \(A\) at all.
\end{definition}

\begin{definition}
  A function is called \emph{injective} if it is also left-unique.\\
  A function is called \emph{surjective} if it is also right-total.\\
  A function is called \emph{bijective} if it is both injective and surjective.
\end{definition}
