\chapter{Tuples}\label{ch:tuples}
\section{Definition}
\begin{definition}
  A tuple is an ordered list of finitely many objects.
  We write tuples surrounded by parentheses, with each element
  separated from the others by a comma.
  A tuple containing \(n\) elements is called an \(n\text{-tuple}\).
  \(2\text{-tuples}\) are sometimes referred to as \emph{pairs}.
\end{definition}
\begin{example}
  [A tuple]
  For example: \((3, 2, 2, 1)\) is the tuple
  containing the numbers three, two, another copy of two,
  and one, in that specific order.
\end{example}

\section{Key Properties}
\begin{itemize}
  \ii{\textbf{Ordered:}} unlike sets, tuples are ordered. That is,
      \((1,2,3)\neq (3,2,1)\),
  because their elements appear in a different order.
  \ii{\textbf{Duplicates:}} tuples may contain elements more than once.
  That is, \((1,2,3) \neq (1,2,2,3,3,3)\), because they contain the same
  elements, but differ in \emph{how often} they occur.
  \ii{\textbf{Finite:}} unlike sets, tuples are always finite.
  They cannot contain an infinite number of elements.
\end{itemize}

\section{Tuple Comparison}
Two tuples \emph{of the same length} may be compared to each other using
\(=, \neq, <, \leq, >, \geq\).
Let \(a = (a_{0}, a_{1}, \dots, a_{n-1}), b = (b_{0}, b_{1}, \dots, b_{n-1})\) be
two tuples of length \(n\). We compare them using the following rules:

\subsection*{Equality}
\begin{itemize}
  \ii{\(a = b\):} if and only if \(\forall i \in \{0, 1, \dots, n-1\}: a_{i} = b_{i}\)
  \ii{\(a \neq b\):} if and only if \(\exists i \in \{0, 1, \dots, n-1\}: a_{i} \neq b_{i}\)
\end{itemize}

\subsection*{Lexicographic Ordering}
\begin{itemize}
  \ii{\(a < b\):} if and only if
        \(\exists k \in \{0, 1, \dots, n-1\}: \forall i < k: a_{i} = b_{i} \land a_{k} < b_{k}\)
  \ii{\(a \leq b\):} if and only if \(a < b \lor a = b\)
  \ii{\(a > b\):} if and only if
        \(\exists k \in \{0, 1, \dots, n-1\}: \forall i < k: a_{i} = b_{i} \land a_{k} > b_{k}\)
  \ii{\(a \geq b\):} if and only if \(a > b \lor a = b\)
\end{itemize}

This is called \emph{lexicographic} ordering, because it is precisely how a
lexicon (or a dictionary) orders words of the same length. Those with the
smaller initial letter are grouped first, then within that group we sort by the
second letter, and so on.

\begin{example}
  \begin{itemize}
    \leavevmode
    \ii{\((1,2,3) < (1,3,2)\)} because \(2 < 3\)
    \ii{\((1,2,3) \geq (1,2,3)\)} because they are equal.
    \ii{\((1,2,3) \neq (4,3,2,1)\)} are not equal, because they differ in
          length. We may not state that one is lesser or greater than the
          other, since they cannot be compared lexicographically.
  \end{itemize}
\end{example}


\section{Tuple Operations}
Much like logical statements, sets have their own associated operations.
In the following list, let \(A, B\) be any two sets.
\begin{itemize}
  \ii{\textbf{Element Access:}}
        \(t[n]\) refers to the \(n\text{-th}\) element of the tuple \(t\).
        We begin indexing elements at \(0\).
        For instance, \((1,2,3)[0] = 1\).
        \begin{remark}
          This is common notation in computer science, hence what we will use for
          purposes of this book. In mathematics, tuples are often indexed starting
          at \(1\) and element access is written is \(t_{n}\) or \(\pi_{n}(t)\).
        \end{remark}
  \ii{\textbf{Concatenation:}}
        \(t \circ u\) is the tuple which contains all elements of \(t\),
        followed by all elements of \(u\), where \(t,u\) are tuples.
  \ii{\textbf{Slicing:}}
        \(t[n:m] \mid m > n\) refers to the sub-tuple of length \(m-n\)
        containing all elements from the \(n\text{-th}\) up to the \(m\text{-th}\)
        element of \(t\), where the \(n\text{-th}\) element is included and the
        \(m\text{-th}\) element is not.
\end{itemize}

\begin{example}
  [Tuple Operations]
  \begin{itemize}
    \leavevmode
    \ii{\((8,12,82,-3,1)[2] = 82\)}
    \ii{\((1,2,3)\circ (4,5,6) = (1,2,3,4,5,6)\)}
    \ii{\((1,2,3,4,5,6)[2:5] = (3,4,5)\)}
\end{itemize}
\end{example}

% TODO: Add exercises and problems
