\chapter{Sets}\label{ch:sets}
\section{Definition}
\begin{definition}
  A set is a collection of elements that does not care about their order or
  multiplicity. We write sets surrounded by curly braces, with each element
  separated from the others by a comma.
\end{definition}
\begin{example}
  For example: \(\{1, 2, 3\}\) is the set
  containing the natural numbers one, two and three. It is equal to
  \(\{3,2,1\}\) and to \(\{1,2,2,3,3,3\}\), since they each contain exactly the
  same elements, only in a different order or multiplicity.
\end{example}

The set \(\NN\) is sometimes understood to include \(0\), sometimes understood
to not include it. This ambiguity is annoying.
In this book, I shall always explicitly use \(\NN_{0}\) to denote the set of
natural numbers \emph{including} zero and \(\NN^{*}\) to denote the set of
natural numbers \emph{without} zero.

\section{\problemhead}
\begin{problem}
  [Example Problem]
  Let \(A=\{0,1,2,3\}\). Is \(0 \in A\)?
  \begin{hint}
    Try reading the notation on sets.
  \end{hint}
  \begin{sol}
    Yes. Zero is an element of the set \(A\).
  \end{sol}
\end{problem}
