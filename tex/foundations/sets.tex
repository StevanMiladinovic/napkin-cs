\chapter{Sets}\label{ch:sets}
\section{Definition}
\begin{definition}
  A set is an unordered collection of distinct objects.
  We write sets surrounded by curly braces, with each element
  separated from the others by a comma.
\end{definition}
\begin{example}
  [A set]
  For example: \(A = \{1, 2, 3\}\) is the set
  containing the numbers one, two and three.
\end{example}

\section{Key Properties}
\begin{itemize}
  \ii{\textbf{Unordered:}} sets are unordered. That is, \(\{1,2,3\}=\{3,2,1\}\),
  because they contain the same elements, only in a different order.
  \ii{\textbf{No duplicates:}} sets only contain each element at most once.
  That is, \(\{1,2,3\}=\{1,2,2,3,3,3\}\), because they contain the same
  elements, only differing in \emph{how often} they occur.
\end{itemize}

\section{Membership}
A set may \emph{contain elements}.
To express that a given element \(x\) is contained within a set \(A\),
we write \(x \in A\). To express that an element \(y\) is \emph{not} contained
within a set \(A\), we write \(y \notin A\).

Two sets \(A, B\) are considered equal, precisely when
\((x \in A) \Leftrightarrow (x \in B)\) for all objects \(x\).

There exists a set that has no elements at all. It is called \emph{the empty
  set} and usually denoted as \(\emptyset\).
\(x \in \emptyset\) is a false statement for every possible object \(x\),
whereas \(x \notin \emptyset\) is true for every object \(x\).

\section{Subsets and Powersets}

Let \(A, B\) be sets. Then \(A\) is called a \emph{subset} of \(B\) if and only if
every element of \(A\) is also in \(B\). We express this by the notation
\(A \subseteq B\). We further introduce the notation \(A \subsetneq B\) to mean
\((A \subseteq B) \land (A \neq B)\). In this case, \(A\) is called a \emph{proper
  subset} of \(B\).

We may equivalently state that \(B\) is a \emph{superset} of \(A\), and express
this as \(B \supseteq A\), or \(B \supsetneq A\) if they are not equal.

If \(A\) is a set, then the set of all subsets of \(A\) is called its
\emph{powerset} \(\PP(A)\).

\begin{example}
  [Subsets, supersets]
  \begin{itemize}
    \leavevmode
    \ii{The empty set is a subset of every set}:
          \(\emptyset \subseteq A\) for every set \(A\)
    \ii{Every non-empty set is a \emph{proper superset} of the empty set}:
          \(A \supsetneq \emptyset\) for every set \(A, A \neq \emptyset\)
    \ii{} \(\{1,2,3\} \subseteq \{1,2,3,4,5\}\)
    \ii{} \(a \in A \Leftrightarrow \{a\} \subseteq A\), though the set containing an object \emph{is a
          different thing than the object itself}
    \ii{} if \(A = \{1,2,3\}\), then
          \(\PP(A) = \{\emptyset, \{1\}, \{2\}, \{3\}, \{1,2\}, \{1,3\}, \{2,3\}, \{1,2,3\}\}\)
  \end{itemize}
\end{example}

\begin{exercise}
  Specify, by writing out all their members in curly braces, the following sets:
  \begin{itemize}
    \ii{\(\PP(\{-3, 15\})\)}
    \ii{\(\PP(\{1\})\)}
    \ii{\(\PP(\emptyset)\)}
  \end{itemize}
\end{exercise}

\section{Quantifiers}

Often times, we would want to express a statement that holds \emph{for every
  element} of a set, or at least for \emph{some elements}.
Or perhaps, we might want to state that a statement is true only for
 \emph{exactly one element} within a set. That's what quantifiers do.

\begin{tcolorbox}[colframe=gray!80, colback=white, boxrule=0.4pt, top=4pt, bottom=4pt]
  \begin{minipage}[t]{\textwidth}
    \centering
    \vspace{-10pt}
    \captionof{table}{Logical Quantifiers}
    \vspace{3pt}
    \begin{tabular}{cc}
      \(\forall\) & ``for all''\\
      \(\exists\) & ``there exists at least one (but possibly many)''\\
      \(\exists\mathpunct{!}\) & ``there exists \emph{exactly} one''
    \end{tabular}
  \end{minipage}
\end{tcolorbox}

 \begin{example}
   [Quantifiers]
 Consider, for instance, the statement: \emph{``Every natural number --- including
 zero ---  may be expressed as the sum of two natural numbers''}.

We would express this using the following notation:
\begin{center}
  \(\forall x \in \NN_0 : x = y + z \qquad y,z\in\NN_0\)
\end{center}

And usually mean to imply that this is a true statement, unless specified otherwise.
\end{example}

\section{Describing Sets}
Sets may be described in a number of ways.
The simplest method --- which we have used thus far --- is by simply listing out
all elements within the set.
However, this is only convenient for small, and only possible at all for
\emph{finite} sets.
Infinite sets exist, however, and must be described using different means.
Consider, for instance, the set \(\NN_{0} = \{0,1,2,3,4,5,\dots\}\),
which contains all \emph{natural numbers}, starting at zero.
From this, we may construct further sets, such as
\(\ZZ = \{x-y \mid x, y \in \NN_0\}\), which is the set of all
\emph{integers}.
\begin{remark}
  This notation reads as ``Let \(\ZZ\) be the set containing all differences
  x-y, where x and y are natural numbers'', or more concisely ``Let \(\ZZ\)
  be the set of all possible differences of natural numbers''. It still includes
  all natural numbers, since \(x - 0 \in \ZZ \forall x \in \NN_0\), so we may
  note that \(\NN_0 \subseteq \ZZ\), but it also contains negative integers, since \(0 - x\)
  is similarly contained.
  So we may note that \(\NN_0 \subseteq \ZZ\).
\end{remark}

From these, we can then further construct the set
\(\QQ = \{\frac{x}{y} \mid x,y \in \ZZ, y \neq 0\}\) of all
\emph{rational numbers}.

Sets are central to virtually all branches of mathematics and theoretical
computer science. They provide the language in which most other concepts are
defined and formalized.

We note at this point, that not every \emph{description} of a set creates
another valid set: consider for instance \(\{A \mid A \notin A\}\), or \emph{the
set of all sets that do not contain themselves}. Such a set cannot exist, as its
construction inevitably leads to a paradox: should it contain itself?

\section{Commonly Used Sets}
Commonly used sets include:
\begin{itemize}
  \ii{\(\NN\):} The set of all \emph{natural numbers}, meaning all the positive integers
        you'd use to count things.
\begin{remark}
  The set \(\NN\) is sometimes understood to include \(0\), sometimes understood
  to not include it. This ambiguity is annoying.
  In this book, I shall always explicitly use \(\NN_{0}\) to denote the set of
  natural numbers \emph{including} zero and \(\NN^{*}\) to denote the set of
  natural numbers \emph{without} zero.
\end{remark}
  \ii{\(\ZZ\):} The set of all \emph{integers}, including the negative integers.
  \ii{\(\QQ\)} The set of all \emph{rational numbers}, which are the ratios of integers,
        ie.\ fractions.
  \ii{\(\RR\):} The set of all \emph{real numbers}, including all those that cannot be
        expressed as fractions, such as \(\sqrt{2}, \pi\) or \(e\)
  \ii{\(\CC\)} The set of all \emph{complex numbers}
\end{itemize}

Each of these sets is also a \emph{proper subset} of the next, that is
\begin{align*}
  \NN \subsetneq \ZZ \subsetneq \QQ \subsetneq \RR \subsetneq \CC
\end{align*}

\section{Set Operations}
Much like logical statements, sets have their own associated operations.
In the following list, let \(A, B\) be any two sets.
\begin{itemize}
  \ii{\textbf{Intersection:}}
        \(A \cap B \defeq \{x \mid x \in A \land x \in B\}\) describes the set of all
        elements that are in \emph{both \(A\) and \(B\)}.
  \ii{\textbf{Union:}}
        \(A \cup B \defeq \{x \mid x \in A \lor x \in B\}\) describes the set of all
        elements that are in \emph{\(A\) or \(B\)} (or in both).
  \ii{\textbf{Difference:}}
        \(A \setminus B \defeq \{x \mid x \in A \land x \notin B\}\) often called \emph{``set
        minus''}, describes the set of all elements that are in \(A\), but
        \emph{not in \(B\)}.
  \ii{\textbf{Symmetric Difference:}}
        \(A \symdif B \defeq \{x \mid x \in A \xor x \in B\}\) the set of all elements in
        \emph{exactly one of} \(A, B\).
  \ii{\textbf{Cartesian Product:}}
        \(A \times B \defeq \{(x, y) \mid x \in A \land y \in B\}\)\\
        Where \((x,y)\) is the \emph{ordered pair} of two elements, one each from \(A, B\).
        We will specify what an \emph{ordered pair} is in \Cref{ch:tuples}.
        To specify the Cartesian product of a set with itself, we use \(A^{2}\), or when
        repeating this operation \(n\) times, \(A^{n}\).
\end{itemize}

\begin{exercise}
  Specify, by writing out all their members in curly braces, the following sets:
  \begin{itemize}
    \ii{\(\{0,1,2,3,4,5\} \cap \{-5,-4,-3,-2,-1,0\}\)}
    \ii{\(\{0,1,2,3,4,5\} \cup \{-5,-4,-3,-2,-1,0\}\)}
    \ii{\(\{0,1,2,3,4,5\} \setminus \{-5,-4,-3,-2,-1,0\}\)}
    \ii{\(\{0,1,2,3,4,5\} \symdif \{-5,-4,-3,-2,-1,0\}\)}
    \ii{\(\{1,2,3\} \times \{x,y,z\}\)}
  \end{itemize}
\end{exercise}

\begin{example}
  [Set Operations]
  \begin{itemize}
    \leavevmode
    \ii{\(\NN^{*} = \NN_{0} \setminus \{0\}\)}
    \ii{\(\RR \cap \ZZ = \ZZ\)}
    \ii{\(\RR^{2} = \RR \times \RR = \{(x, y) \mid x, y \in \RR\}\)}
      \\is the set of all points in the flat plane, and more generally
    \ii{\(\RR^{n} = \RR \times \cdots \times \RR = \{(x_{1}, \ldots, x_{n}) \mid x_{i} \in \RR\}\)}
      \\is the set of all points in \(n\text{-dimensional}\) space.
\end{itemize}
\end{example}
