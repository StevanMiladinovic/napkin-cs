\chapter{Sets}\label{ch:sets}
\section{Definition}
\begin{definition}
  A set is an unordered collection of distinct objects.
  We write sets surrounded by curly braces, with each element
  separated from the others by a comma.
\end{definition}
\begin{example}
  [A set]
  For example: \(A = \{1, 2, 3\}\) is the set
  containing the numbers one, two and three.
\end{example}

\section{Key Properties}
\begin{itemize}
  \ii{\textbf{Unordered:}} sets are unordered. That is, \(\{1,2,3\}=\{3,2,1\}\),
  because they contain the same elements, only in a different order.
  \ii{\textbf{No duplicates:}} sets only contain each element at most once.
  That is, \(\{1,2,3\}=\{1,2,2,3,3,3\}\), because they contain the same
  elements, only differing in \emph{how often} they occur.
\end{itemize}

\section{Membership}
A set may \emph{contain elements}.
To express that a given element \(x\) is contained within a set \(A\),
we write \(x \in A\). To express that an element \(y\) is \emph{not} contained
within a set \(A\), we write \(y \notin A\).

Two sets \(A, B\) are considered equal, precisely when
\((x \in A) \Leftrightarrow (x \in B)\) for all objects \(x\).

There exists a set that has no elements at all. It is called \emph{the empty
  set} and usually denoted as \(\emptyset\).
\(x \in \emptyset\) is a false statement for every possible object \(x\),
whereas \(x \notin \emptyset\) is true for every object \(x\).

\section{Describing Sets}
Sets may be described in a number of ways.
The simplest method --- which we have used thus far --- is by simply listing out
all elements within the set.
However, this is only convenient for small, and only possible at all for
\emph{finite} sets.
Infinite sets exist, however, and must be described using different means.
Consider, for instance, the set \(\mathbb{N}_{0} = \{0,1,2,3,4,5,\dots\}\),
which contains all \emph{natural numbers}, starting at zero.
From this, we may construct further sets, such as
\(\mathbb{Z} = \{x-y \mid x, y \in \mathbb{N}_0\}\), which is the set of all
\emph{integers}.
From these, we can then further construct the set
\(\mathbb{Q} = \{\frac{x}{y} \mid x,y \in \mathbb{Z}, y \neq 0\}\) of all
\emph{rational numbers}.

Sets are central to virtually all branches of mathematics and theoretical
computer science. They provide the language in which most other concepts are
defined and formalized.

We note at this point, that not every \emph{description} of a set creates
another valid set: consider for instance \(\{A \mid A \notin A\}\), or \emph{the
set of all sets that do not contain themselves}. Such a set cannot exist, as its
construction inevitably leads to a paradox.

\section{Subsets and Powersets}

Let \(A, B\) be sets. Then \(A\) is called a \emph{subset} of \(B\) if and only if
every element of \(A\) is also in \(B\). We express this by the notation
\(A \subseteq B\). We further introduce the notation \(A \subsetneq B\) to mean
\((A \subseteq B) \land (A \neq B)\). In this case, \(A\) is called a \emph{proper
  subset} of \(B\).

We may equivalently state that \(B\) is a \emph{superset} of \(A\), and express
this as \(B \supseteq A\), or \(B \supsetneq A\) if they are not equal.

\begin{example}
  [Subsets, supersets]
  \begin{itemize}
    \leavevmode
    \ii{The empty set is a subset of every set}:
          \(\emptyset \subseteq A\) for every set \(A\)
    \ii{Every non-empty set is a \emph{proper superset} of the empty set}:
          \(A \supsetneq \emptyset\) for every set \(A, A\neq\emptyset\)
    \ii{} \(\{1,2,3\} \subseteq \{1,2,3,4,5\}\)
    \ii{} \(a \in A \Leftrightarrow \{a\} \subseteq A\)
  \end{itemize}
\end{example}

If \(A\) is a set, then the set of all subsets of \(A\) is called its
\emph{powerset} \(\PP(A)\).

\section{Quantifiers}

\section{Set Operations}

\section{Commonly Used Sets}
Commonly used sets include:
\begin{itemize}
  \ii{\(\mathbb{N}\):} The set of all \emph{natural numbers}
\begin{remark}
  The set \(\NN\) is sometimes understood to include \(0\), sometimes understood
  to not include it. This ambiguity is annoying.
  In this book, I shall always explicitly use \(\NN_{0}\) to denote the set of
  natural numbers \emph{including} zero and \(\NN^{*}\) to denote the set of
  natural numbers \emph{without} zero.
\end{remark}
  \ii{\(\mathbb{Z}\):} The set of all \emph{integers}
  \ii{\(\mathbb{Q}\):} The set of all \emph{rational numbers}
  \ii{\(\mathbb{R}\):} The set of all \emph{real numbers}
  \ii{\(\mathbb{C}\):} The set of all \emph{complex numbers}
\end{itemize}

\section{\problemhead}
\begin{problem}
  [Example Problem]
  Let \(A=\{0,1,2,3\}\). Is \(0 \in A\)?
  \begin{hint}
    Try reading the notation on sets.
  \end{hint}
  \begin{sol}
    Yes. Zero is an element of the set \(A\).
  \end{sol}
\end{problem}
